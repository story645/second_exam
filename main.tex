\documentclass[letterpaper,onecolumn,titlepage]{Ythesis}

\usepackage{graphicx}
\graphicspath{{sections/figs/}{.figs/}}

\usepackage[backend=bibtex, style=numeric-comp]{biblatex}
\bibliography{main}

\usepackage{subfiles}
\usepackage{url}
\usepackage{amsmath}



\title{Visualizing Functional Data}
\author{Hannah Aizenman}
\committee{ Dr. Michael Grossberg(Advisor), Robert Haralick, Huy Vo}
\submitted{}
\abstract{}

\begin{document}
\makefrontmatter

\section{Introduction}

High dimensional data is catch-all term, typically used to describe data that's highly
multivariate or data that's structurally complex, or even data that simply has
a large number of observations. The lack of specificity of the phrase
`high-demensional` can make it difficult to parse out which high-dimensional
analysis and visaualization technique is suited to the research question at
hand. To answer that question, it's crucial to first have a set of working
definitions for discussing the data:

\begin{definition}
\item[measurements] criteria based assignment of numbers to objects or events \cite{grimm1993}
\item[variables] ``any characteristics, number, or quantity that can be measured
  or counted''\cite{ABS}

\item[field] dataset type where each cell has associated information \cite{munzner14ii}    
\end{defintion}




\subfile{sections/independent.tex}

\section{Conclusion}
\label{sec:conclusion}

\begin{tabular}{|r|r|r|r|}
  \hline
              & univariate & bivariate & multivariate\\
  \hline
   0D         &            &           &             \\
  \hline
   1D         &            &           &             \\
   \hline
   ND         &            &           &             \\
\end{tabular}

\pagebreak
\printbibliography
\end{document}
