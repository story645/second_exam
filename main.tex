\documentclass[letterpaper,onecolumn,titlepage]{Ythesis}

\usepackage{graphicx}
\graphicspath{{sections/figs/}{.figs/}}

\usepackage[backend=bibtex, style=ieee]{biblatex}
\bibliography{main}

\usepackage{subfiles}
\usepackage{url}
\usepackage{amsmath}



\title{Visualizing Multivariate PDFs}
\author{Hannah Aizenman}
\committee{ Dr. Michael Grossberg(Advisor), Robert Haralick, Huy Vo}
\submitted{}
\abstract{}

\begin{document}
\makefrontmatter

\section{Introduction}
Tukey lays out the case for exploratory data analysis when
he says ``The picturing of data must be sensitive, not only to the multiple
hypotheses we hold, but to the many more we have not yet thought of, regard as unlikely or
think impossible'' \cite{tukey1975}. He advocates breaking from a deterministic $$y=mx+b$$
best fit line approach where the b is treated as an afterthough to a more
nuanced
\begin{align*}
  given = fit PLUS residual
\end{align*}
explicitly to bring attention to the residuals. Tukey is arguing that it's not
good enough to just look at the path of the data but also at the variation in
the path. In some sense this is the fundemental disctinction between
confirmatory data analysis and exploratary data analysis; whereas the former
intends to display a picture that supports a hypothesis (such as that best fit
line that shows the direction of the data), the latter's purpose is to
facilitate investigation of the data (and the hypothesis). To explore data, it
is crucial to understand the shape of the data, which can be done using a
number of exploratory visualzation techniques. Because these techniques are
highly dependent on the structure of the data, this paper explores techniques suitable to different classes of data-seperable, non-seperable, and
possibly seperable but interdependent-and then uses this framework to evaluate
distrubitional visualziation in recent climate visualization toolkits. 

\subfile{sections/seperable.tex}
%%history of the boxplot and the like
\section{Non-Seperable Dimensions}
%%using machine learning to find seperable distributions in data/
%%PCA, Clustering,ICA, Linear Manifold, etc...
\section{Interdependent but Important Dimensions}
%% Spatio Temporal Probabilitues - hurrican tracks yield curves 
\section{Climate Visualization Toolkits}

\section{Conclusion}
\label{sec:conclusion}


\pagebreak
\printbibliography
\end{document}
