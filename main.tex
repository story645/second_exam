\documentclass[letterpaper,onecolumn,titlepage]{Ythesis}

\usepackage{graphicx}
\graphicspath{{sections/figs/}{.figs/}}

\usepackage[backend=bibtex, style=numeric-comp]{biblatex}
\bibliography{main}

\usepackage{subfiles}
\usepackage{url}
\usepackage{amsmath}



\title{Visualizing Functional Data}
\author{Hannah Aizenman}
\committee{ Dr. Michael Grossberg(Advisor), Robert Haralick, Huy Vo}
\submitted{}
\abstract{There are an almost overwhelming number of machine learning and
  visualization techniques, but each technique makes some underlying
  assumptions about how the data is structured. While sometimes the
  observations are completely independent from each other, in many instances
  the observations are sampled from 1D continuum like time or 2d surface such
  as space.This survey uses functional data analysis to classify visualizations based on
  the number of variables, whether the variables are independent, and how the
  observations are related to each other}
\begin{document}
\makefrontmatter

\section{Introduction}

High dimensional data is catch-all term, typically used to describe data that's highly
multivariate or data that's structurally complex, or even data that simply has
a large number of observations. The lack of specificity of the phrase
`high-demensional` can make it difficult to parse out which high-dimensional
analysis and visaualization technique is suited to the research question at
hand. Datasets inherently have two components, the variables and the
observations of those variables. These oberservations are sometimes dependent
on information rich parameters such as time and space, and often it is as
important to visualize these underlying parametric dependencys as the
relationships between the variables in given observations. Functional data
analysis provides a language for formalizing the structure of the data in such a way that it clarifies the
possible interactions in the data.

%%paragraph on functional data analysis
The underlying assumption of functional data analysis is that while sometimes
samples are discrete, often they are drawn from a continuos sample of
obersvations \cite{ramsey, hook, meuller}. Independent observations
by definition rely on zero paramters, whereas dependent observations can be
paramterized on one dimension (such as as time), or multiple (such as time,
latitude, longitude, altitude). %%throw some math here to transition to my table


\begin{table}
\begin{tabular}
   & univarate   & bivariate                & multivariate \\
0 parameter & RV_i         & v0_i, v1_i               & RV0_i, \dots, RVn_i \\
1 paramter & RV_i(x_i)     & RV0_i(x_i), RV1_i(x_i)    & RV0_i(x_i), \dots, RVn_i(x_i)\\
N parameters & RV_i(x0_i,\dots,xn_i)     & RV0_i(x0_i,\dots,xn_i), RV1_i(x0_i,\dots,xn_i)    & RV0_i(x0_i,\dots,xn_i), \dots, RVn_i(x0_i,\dots,xn_i)\\
\end{tabular}
\caption{Each $vi$ is a distinct variable such as precipitation or temperature,
  each $xi$ is a distinct parameter such as time or space, and i is the
  observation identifier.}
\end{table}

The variables $v0,\dots, vn$ can also potentially be dependent on each other,
adding in a third level of interactions that a visualization may want to
caputure. While this survey focuses on data structured in a linear, tabular, or
data cube way, it could potentially be expanded to network data where the xi
parameters represent edges in the graph. 


\subfile{sections/independent.tex}

\section{Conclusion}
\label{sec:conclusion}

\begin{tabular}{|r|r|r|r|}
  \hline
              & univariate & bivariate & multivariate\\
  \hline
   0D         &            &           &             \\
  \hline
   1D         &            &           &             \\
   \hline
   ND         &            &           &             \\
\end{tabular}

\pagebreak
\printbibliography
\end{document}
