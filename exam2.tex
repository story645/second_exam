\documentclass[letterpaper,onecolumn,titlepage]{Ythesis}
\usepackage{import}
\usepackage{siunitx}
\usepackage{amsmath}
\usepackage{url}
\sisetup{detect-weight=true, detect-family=true}
\newcommand{\micron}[1] {\SI{#1}{\micro\meter}}

%\usepackage{setspace}
%\onehalfspace

\title{Visualizing Multivariate PDFs}
\author{Hannah Aizenman}
\committee{ Dr. Michael Grossberg(Advisor), Robert Haralick, Huy Vo}
\submitted{}
\abstract{
  Problem:
  Importance:
  Solution:
}

\begin{document}
\makefrontmatter

\section{Introduction}

Munzner \cite{Munzner14ii} defines a time-varying dataset as one in which time is an intrinsic attribute of how the various observations are measured; for example recording snow fall every 3 hours or stock prices at the beginning or end of day. She contrasts this with datasets that are amassed over time where the length of the record does not intrinsically mean the data is time varying; for example [insert something that isn't horse racing] 

Often the observations in these time-varying datasets are also spatially varying, spreading over the earth or the brain or another variable space where all the observations in the high dimensional space are of the same unit. %%note: must get some reference/notation/way to discuss this

This poses a difficult visualization task because the researcher is trying to capture multiple levels of interaction:
\begin{enumerate}
	\item intraobservational: snowfall in New York and snowfall in New Jersey on August 9th 2015)
	\item interobservational: global snowfall on August 9th and August 10th
	\item a mixture thereof: does snowfall in New York on the 9th affect snowfall in New Jersey on the 10th?
\end{enumerate}

\section{Seperable Dimensions}
\import{sections/}{seperable.tex}
%%history of the boxplot and the like
\section{Non-Seperable Dimensions}
%%using machine learning to find seperable distributions in data/
%%PCA, Clustering,ICA, Linear Manifold, etc...
\section{Interdependent but Important Dimensions}
%% Spatio Temporal Probabilitues - hurrican tracks yield curves 
\section{Climate Visualization Toolkits}

\section{Conclusion}
\label{sec:conclusion}


\pagebreak
\bibliographystyle{abbrv}
\bibliography{exam2}

\end{document}
