==\documentclass[letterpaper,onecolumn,titlepage]{Ythesis}

\usepackage{graphicx}
\graphicspath{{sections/figs/}{.figs/}}

\usepackage[backend=bibtex, style=numeric-comp]{biblatex}
\bibliography{main}

\usepackage{subfiles}
\usepackage{url}
\usepackage{amsmath}



\title{Visualizing Conditional Relationshops}
\author{Hannah Aizenman}
\committee{Dr. Michael Grossberg(Advisor), Dr. Robert Haralick, Dr. Huy Vo}
\submitted{}
\abstract{The context surrounding data is often as important as the
  measurements themselves. Income data provides a lens into demographic
  inequalities when seperated by race and gender; temperature data indicates
  seasons when plotted by time. These are relatively simple relationships, but
 many datasets are embedded with rather complex conditional relationships. This
 survey gives an overview of techniques showing conditional dependencies
 amongst distributions and densities. 
}
\begin{document}
\makefrontmatter

\section{Introduction}
\label{sec:introduction}
p(X|Y) is normalized so that holding Y fixed and summing over X it sums to 1 oh
actually they write it P(X|Y)

P is discrete and p is density continuous

P is a probability distribution function p is a propability density function

Sheldon M. Ross .... intro to prob models
Discrete Random Variable Continuous Random Variables
Categorical Parameters
Continuous Parameters

Continuous: time, space, time+space


The underlying assumption of functional data analysis is that while sometimes
samples are discrete, often they are drawn from a continuos sample of
obersvations \cite{ramsey2006, ramsey2002, muller2006}. That continuous sample is often over
space or time or another attribute on which the observation is conditional. In
functional data analysis,

\begin{table}%%use conditional notation since it's the same
\begin{tabular}
                & Distribution & Density 
   Distribution &
   Density      & 
\end{tabular}

\begin{table}%%use conditional notation since it's the same
\begin{tabular}
   & univarate   & bivariate                & multivariate \\
0 & RV         & RV_0, RV_1               & RV_0, \dots, RV_n \\
1 & RV(x)     & RV_0(x), RV_1(x)    & RV_0(x), \dots, RV_n(x)\\
N & RV(x_0,\dots,x_n)     & RV_0(x_0,\dots,x_n), RV_1(x_0,\dots,x_n)    & RV_0(x_0,\dots,xn_i), \dots, RV_n(x_0,\dots,x_n)\\
\end{tabular}

\caption{Each RV is a distinct variable such as precipitation or temperature,
  each x is a parameter, such as time or space, on which RV is conditionally
  dependent. Each cell in the table is distinct observation in the dataset}
\end{table}

The variables $v0,\dots, vn$ can also potentially be dependent on each other,
adding in a third level of interactions that a visualization may want to
caputure. While this survey focuses on data structured in a linear, tabular, or
data cube way\cite{munzner data structure}, it could potentially be expanded to network data where the xi
parameters represent edges in the graph. 


\subfilt{sections/problem.tex}
\subfile{sections/independent.tex}

\section{Conclusion}
\label{sec:conclusion}

\pagebreak
\printbibliography
\end{document}
