\documentclass[letterpaper,onecolumn,titlepage]{Ythesis}

\usepackage{graphicx}
\graphicspath{{sections/figs/}{.figs/}}

\usepackage[backend=bibtex, style=numeric-comp]{biblatex}
\bibliography{main}

\usepackage{subfiles}
\usepackage{url}
\usepackage{amsmath}



\title{Visualizing Conditional Relationships}
\author{Hannah Aizenman}
\committee{Dr. Michael Grossberg(Advisor), Dr. Robert Haralick, Dr. Lev Manovich, Dr. Huy Vo}
\submitted{}
\abstract{Many datasets have qualitative and quantitative variables and  also
  often have functional attributes like time and location and visualizations
  provide views on a subset of these variables and how they relate to each
  other. For example a timeseries plot will keep the link between the variable
  and time but lose the information encoded in the space variable. These
  relationships can be modeled as conditional probabilities  such as P(V|t),
  and visualizations can be organized based on which of these probabilities are
  implicitly preserved in the visualization techniques. Modern visualizations
  in particular try to tackle the difficulties of displaying these functional
  relationships as the number of variables and the number of functional
  dependencies (for example lat, lon, height, time, and type) increase. }
\begin{document}
\makefrontmatter

\section{Introduction}
\label{sec:introduction}
\begin{figure}
  \includegraphics[width=\textwidth]{chart_chooser.png}
  \caption{From Bertin's Semiology of Graphics \cite{bertin_semiology_2011}, this diagram
    illustrates how there are many ways to display the same information about the 1954 French
  workforce.}
  \label{fig:chart_chooser}
\end{figure}

There are almost an overwhelming number of visualizations, and no real correct
choice for any given situation. As figure~\ref{fig:chart_chooser} shows, often
there isn't even a single standout chart for a single dataset; instead the
choice of figure is dependent on the relationships the researcher would like to
extract from the dataset. Is it important to know how many people are in each
department? A bar chart is suited for this, but this data has that information
segmented into sectors - should it be combined or stay separated? Is the
spatial distribution of the workforce something that should be preserved? If
so, how should the sector information be retained? Sectors and space can both
conditional attributes of the data, yielding information on how the workforce
varies when disaggregated into these smaller blocks.

Quite often, it is this disaggregation that is interesting to researchers
because it points to multiple factors that can contribute to the measured
outcome. The difficulty is in figuring out how to visualize this
disaggregation, especially as the data structure grows more complex. Cross
tabulation tables work for a small number of categorical variables, but are
pushed to the limits at n=3 or 4. And most data is not solely categorical,
instead often having quantitative densities of one sort or another. A complex but
important form of quantitative densities are functional densities, where they are drawn from a continuous sample of observations \cite{ramsay_functional_2006, muller_functional_2006}. That continuous sample is often over space or time or another parameter or combination thereof, and often it is
important to preserve those functional dependencies in the visualization. For
example, for storm modeling it is important to show the expected time and
location of the rainfall. This survey presents a sampling of visualization
techniques that attempt to illustrate these conditional and functional
relationships within a dataset. 



\begin{table}
\begin{center}
\begin{tabular}{lll}
   Variable Type& Distribution           & Density\\ 
   Distribution & $P(X\|Y),P(A,B,C|E,F,G)$  &  \\
   Density      &                        & $p(X\|Y), p(A,B,C\|E,F,G)$\\
\end{tabular}
\caption{The rows represent the known distributions and densities while the
  columns are the distributions and densities that the visualization aims to derive.}
  \end{center}
\end{table}

While this survey focuses on data structured in a linear, tabular, or
data cube way \cite{munzner_what:_2014}, it could potentially be expanded to network data.


\subfile{sections/distdist.tex}
\subfile{sections/distdens.tex}
\subfile{sections/densdist.tex}
\subfile{sections/densdens.tex}

\section{Conclusion}
\label{sec:conclusion}
When trying to choose a visualization technique for a problem, it is often
crucial to interrogate the questions being asked of the data. What
relationships are important to preserve and which can be discarded? While there
are a what feels like infinite number of visualization techniques, this
visualization task can be rather complicated, especially as the number of
variables or dependencies increases. Interactive tools can help, but most are limited to showing pairwise or a small number of connected interactions. Machine learning can somewhat be used
to explore which variables and dimensions, and therefore which relationships, are worth
exploring further, but often the results of those algorithms need to
be translated to the visual space and those visualizations need to be
understandable. 


\printbibliography
\end{document}
