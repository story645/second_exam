==\documentclass[letterpaper,onecolumn,titlepage]{Ythesis}

\usepackage{graphicx}
\graphicspath{{sections/figs/}{.figs/}}

\usepackage[backend=bibtex, style=numeric-comp]{biblatex}
\bibliography{main}

\usepackage{subfiles}
\usepackage{url}
\usepackage{amsmath}



\title{Visualizing Functional Data}
\author{Hannah Aizenman}
\committee{Dr. Michael Grossberg(Advisor), Dr. Robert Haralick, Dr. Huy Vo}
\submitted{}
\abstract{There are an almost overwhelming number of machine learning and
  visualization techniques, but each technique makes some underlying
  assumptions about how the data is structured. While sometimes the
  observations are completely independent from each other, in many instances
  the observations are sampled from 1D continuum like time or 2d surface such
  as space.This survey uses functional data analysis to classify visualizations
  based on the number of variables in and level of paramterization of the data.}
\begin{document}
\makefrontmatter

\subfile{sections/intro.tex}
\subfilt{sections/problem.tex}
\subfile{sections/independent.tex}

\section{Conclusion}
\label{sec:conclusion}

\begin{tabular}{|r|r|r|r|}
  \hline
              & univariate & bivariate & multivariate\\
  \hline
   0D         &            &           &             \\
  \hline
   1D         &            &           &             \\
   \hline
   ND         &            &           &             \\
\end{tabular}

\pagebreak
\printbibliography
\end{document}
