\documentclass[../main.text]{subfiles}
\begin{document}

The underlying assumption of functional data analysis is that while sometimes
samples are discrete, often they are drawn from a continuos sample of
obersvations \cite{ramsey2006, ramsey2002, muller2006}. That continuous sample is often over
space or time or another attribute on which the observation is conditional. In
functional data analysis,




\begin{table}%%use conditional notation since it's the same
\begin{tabular}
   & univarate   & bivariate                & multivariate \\
0 & RV         & RV_0, RV_1               & RV_0, \dots, RV_n \\
1 & RV(x)     & RV_0(x), RV_1(x)    & RV_0(x), \dots, RV_n(x)\\
N & RV(x_0,\dots,x_n)     & RV_0(x_0,\dots,x_n), RV_1(x_0,\dots,x_n)    & RV_0(x_0,\dots,xn_i), \dots, RV_n(x_0,\dots,x_n)\\
\end{tabular}
\caption{Each RV is a distinct variable such as precipitation or temperature,
  each x is a parameter, such as time or space, on which RV is conditionally
  dependent. Each cell in the table is distinct observation in the dataset}
\end{table}

The variables $v0,\dots, vn$ can also potentially be dependent on each other,
adding in a third level of interactions that a visualization may want to
caputure. While this survey focuses on data structured in a linear, tabular, or
data cube way, it could potentially be expanded to network data where the xi
parameters represent edges in the graph. 

\end{document}
