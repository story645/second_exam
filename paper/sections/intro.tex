\documentclass[../main.text]{subfiles}
\begin{document}

A high resolution spatio-temporal temperature model and the academic
transcripts of all students at The Graduate Center are both high dimensional
data sets, but they pose intrinsically different data visualization
challenges. The former necessitates visualization techniques that preserve the
spatio-temporal structure, whereas the latter requires a technique for
highlighting the important information in a sea of variables. 



%% include retinal variables figure
Bertin's retinal variables codified the language of visualizaton. As
figure~\ref{fig:bertin} describes, qualitiave aspects of a visual symbol -
it's shape, hue, intensity, texture-are varied to display qualitative
differences in the data, and the quantatitive attributes - size, color value -
are manipulated to show quantatiative differences in the data
\cite{bertin1983, krygier2016}. Figure~\ref{fig:bertin} illustrates the
underlying implict distinction between variables and the components of the data
that inform the structure of the data (the geographic plane or the time series
line). In this system, the data's internal structure informs whether it will be
represented as a point, a line, or a plane. While more structural components
can be visualized using shape, size, hue, etc. there's no assumed
representation outside what can be easily drawn on paper.  



\cite{wilkinson2006}
