Some research institutions provide a tool to explore data they created, such as the \cite{src:esrlpsd}, but since all these tools are site specific, a researcher would either have to use the datasets on the website or submit their datasets for uploading 
(assuming their dataset meets the submission criteria for an aggregation site). A scientist also has the option of using the CDAT suite of libraries, developed by the \cite{WilliamsEtAl13}, GrADS, \cite{src:grads},and Ferret, \cite{src:HankinEtAl96}. The major critique with all of these though is that they mostly render the data as is, with limited support for statistical analysis and the application of typical meterological analysis. Support for even common machine learning techniques often requires extending the library; therefore for the purpose of this discussion it is somewhat useful to treat these tools more as plotting libraries than visualization systems. %%somewhere scope out/differentiate between plotting and viz
