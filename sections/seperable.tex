\documentclass[../main.tex]{subfiles}

\begin{document}
\section{Seperable Dimensions}
\label{sec:sepdim}

While bar graphs and histograms start giving a sense of the variability in
data, they are limited to showing the dispersion of the data only in 1
dimension; namely the measurements themselves. In their survey of the history
of the boxplot \cite{wickham2011} Wickham and Stryjewski trace the evolution of
the boxplot from a difinitive box to amorphous envelopes.  They start by laying out
the core statistics of a boxplot: a box centered at the median (or mean) and
bound by the upper and lower quartiles (termed the hinges), and two whiskers which represent the
extremes (often 1.5 the outer quartiles, sometimes the $\alpha$ bounds). Sometimes there's
also a dot to signify outliers, as seen in \ref{fig:boxplot}, but one of
the strengths of the box plot has been in it's flexibility. 

\subsection{Discrete observations}

\begin{figure}
\includegraphics{boxplot}
\caption{Tukey's 1977 example of the box plot (exhibit 6 of chapter 2 from
Exploratory Data Analytics\cite{tukey1977}) illustrates how the box plot shows
the distribution of topology of states and
the distribution of volcanos. The strength of the box and whisker is that the
reader can quickly compare the two distributions and learn, for example, that
volcanos tend to span a shorter range of average heights, but that the extremes
are much further from the center. Essentially, the distributions act as
references for the other displayed distributions.}
\label{fig:boxplot}
\end{figure}

In Tukey's introduction to the boxplot \cite{tukey1977}, shown in
Figure~\ref{fig:boxplot}, the box and whisker plot takes the distributional information of a histogram and encodes the state heights between the
25th and 75th percentiles as a box, the median height as a line cutting across
the heights, and the extreme state heights as lines coming out of the
boxes. While a histogram would also show that the distribution of state heights has a fat right tail,
the power of the box and whisker plot is in encoding this information such that
it's trivial to compare to it's neighboring distribution of volcano
heights because they can share the Y axis. As the number of distributions grow,
this technique becomes invaluable because the number of histograms that can be
placed next to each other or overlayed as pdfs can quickly become
unweildy.  Many others authors have kept the basic structure of the boxplot,
but exploited the structure to convey more information about the data. Authors
used different quantile levels \cite{hyndman1996}s or measures of outliers
\cite{frigge1989}, or ways of identifying outliers\cite{carter2009, schwertman2004, schwertman2007}, or otherwise incorporated skewness,
kurtosis, and other descriptive distributional statistics \cite{kim2004, hubert2008, marmolejo2015}.

\subsubsection{Incoporating Density}

\begin{figure}
%%\includegraphic{Mcgill} #annottate figure with highlighted line for the notches!
\label{fig:notched}
\end{figure}
The boxplot was then expanded out from descriptive statistics to facilitating
inferential deduction through the notched boxplot. \cite{mcgill1978} The notched boxplot
arose as an expansion of indicating sample size through the width of the box,
which is a technique that only works well for relatively small sample sizes (N
less than 1000), at larger scales the widths are about equal. Instead, notches, as seen
in Figure~\ref{fig:notched}, are used to indicate the confidence level of the
median for that box. The medians are statistically different from each other at
the 95\% confidence level if the notches do not overlap. In
Figure~\ref{fig:notched}, all the notches overlap %%annotate figure with what
%%overlapping region means
and so the difference in telephone bills is not significant, but there's a
statistically significant difference between the 11-15 and over 15 groups. Notches may extend
beyond the box edges, and sometimes the width technique is incorporated back
into the notched boxpplot such that the confidence can be grounded in sample
size. The notches in Figure~\ref{fig:notched} are computed assuming a gaussian
distribution, but many authors \cite{x,y,z} discuss ways to compute notches for
non-gaussian distributions. 

\begin{figure}
%%two figs a and b with a shared caption...
%%\includegraphics{benjamini}
\label{fig:histplot}
\label{fig:vase}
\label{fig:violin}
\end{figure}

Because the box-plot does not give any indication of how data is distributed
inside the box or along the whiskers, the histplot and vaseplot were introduced to indicate
the density of the data\cite{benjamini1988}. As shown in
Figure~\ref{fig:histplot} the histplot seperates the box
into  5 bins and scales the width of that section of the box based on the
number of values in that bin; it's essentially a rotated histogram. The
vaseplot, shown in Figure~\ref{fig:vase} removes the bins and is instead based
on the estimated density at that point in the data. The specific shape is
dependent on what sort of density estimation method is used, such as kernal
estimation or nearest neighbor estimation \cite{chambers1983}. The
violinplot \cite{hintz1998} expands this concept further by representing the
width using the density-trace method (which is a probability density estimation
via moving windowed average) proposed by Chambers et. al \cite{chambers1983},
as seen in Figure~\ref{fig:violinplot}


\subsubsection{2D Distributions}
\begin{figure}
%%\includegraphics{bagplot}
\end{figure}
The boxplot is inherently limited to 1D data; the bagplot was introduced to
visualize 2D and higher data \cite{rousseeuw1999}
Bivariate HDR plot \cite{hyndman1996} \textbf{expand with discussion of boxplot/HDR plot}

\subsection{Functional `observations`}

While boxplots were designed to visualize discrete observations, often there
are no discrete observations and instead the `observation` is a timeseries
curve (where the connections are as important as the individual points) or
spatial region. Understanding the distributions of functional observations
often relies first ordering the functions such that there's some measure of how
close these functions are to each other--functional depth\cite{febrero2007},
bivariate score depth\cite{hyndman2009}, and
bivariate kernel density estimation\cite{scott1992} are some examples. This
ordering then yields functional analogs to median and other percentile
metrics. For small values, the ordered curves can then be colored based on the
ordering metric \cite{hyndman2009}, but this method is unweildy for larger
ensembles.

\begin{figure}
%%\includegraphics{funcbaghdr}
\label{fig:funcbag}
\end{figure}
Instead, Hyndmand and Shang proposed functional versions of the bag and HDR
plots\cite{hyndman2009}, as shown in Figure~\ref{fig:funcbag}. In these graphs,
what was traditionally the box part of a box plot is now a region bounded by
the curves (obtained by using one of the previously mentioned ordering
metrics). The median curve is defined as a blackline and the exceptional
outlines are shown. By using regions defined by the curves, the functional bag
and HDR plots display the variance in the ensembles while simulatanously
flattening the similaraties so that the outliers are strongly highlighted. The
HDR method better highlights outliers closer to the median, whereas the bagplot
method better highlights the outliers outside the envelopes of functions. 



The functional boxplot was developed to describe the distributional nature of
functional data\cite{sun2011}. \textbf{expand this section with discusson of Sun's paper}


\begin{figure}
  \includegraphics[width=.75\linewidth]{contour_weather}
  \caption{}
\label{fig:countour}
\end{figure}
While functional bag, box, and HDR plots all show outliers, each method is most
successful visualizing a specific type of outlier. Countour
boxplots\cite{whitaker2013} aim instead to capture the variablity of the
uncertainity via a band depth ordering metric. Each ensemble members band depth
is computed as sum of the probabilities that the observations in any given
ensemble fall within the max\-min envelope defined by any two other
ensembles. The bands are then sorted by band depth such that the median is the
ensemble with about 50\% of its members withen all envelopes formed by other
bands (so most centered). Outer bands are chosen according to the task at
hand. In \cite{whitaker2013}, they apply the countour boxplots to temperature
visualizations. The authors argue that countour boxplots are an improved visual
idiom over the traditional plots seen in Figure~\ref{fig:spag} because
as seen in , the spagehetti plots get nosy as the number of plots increase
and so it's hard to tease out specific patterns. Instead in Figure~\ref{fig:countour}, the authors
remove most of the bands and instead visualize an outlier envelope(light gray)
and a more central envelope(dark gray). It retains much of the information of
the spagethhit, but removes the visual noise of the lines.

\end{document}

