While bar graphs and histograms start giving a sense of the variability in
data, they are limited to showing the dispersion of the data only in 1
dimension; namely the measureents themselves. To explore more dimensions,
Tukey introduced the box and whiskers plot \cite{tukey1977e}. The box and
whisker plot takes the distributional information of a
histogram and encodes the summary statistics in a box and outlier information as
whiskers coming out of the box. While this is inherently useful, the
information could have just as easily been shown in a histogram. The power of
the boxplot is in that it facilitates the comparison of multiple distributions:
\begin{figure}
%% side by side  boxplot comparsions of random data?
\end{figure}

Where the box and whisker plot really shines is in comparing the extreme
values because they are represented as lines from the center and therefore the
length of the two lines can be compared \cite{tukey1977e}.  Also, by
translating the distribution statistics to a visual representation, the
outliers are proportioanlly scaled to the mean and standard, which gives a
sense of how the summary stats relate to each other that is sometimes lost in
numerical calculations when the numbers are small or close to each
other. %%probably needs a citation

In their survey of the history of the
boxplot \cite{Wickham2011} Wickham and Stryjewski trace the evolution of the
boxplot from a difinitive box to amorphous envelopes.  They start by laying out
the core statistics of a boxplot: a box centered at the median (or mean) and
bound by the upper and lower quartiles, and two whiskers which represent the extremes
(often 1.5 the outer quartiles, sometimes the \alpha bounds). Sometimes there's
also a dot to signify outliers, as seen in \ref{imaginary figure}, but one of
the strengths of the box plot has been in it's flexibility. Many authors have
kept the basic structure, but used different quantile \cite{Hyndman}s or
measures of extremes \cite{Frigge, carter} or used assymetric whiskers \cite{Rousseuw}
or otherwise exploited the structure to incorporate skewness, kurtosis, and
other descriptive distributional statsics \cite{ Aslam, choon, Marmelejo}
%find papers not in Wickham

%notched boxplot





Inferential information was incorporated through the notched boxplot. \cite{McGill} % find/print this
% paper variations
In this plot 








%letter value plots

%bagplots-> 
Countour Boxplots \cite{Whitaker2013} are an extension of boxplots that tries to better encapsulate outliers. Box plots inherently clip the uncertainity they show to some upper and lower band, 
creating a somewhat bounding envelope for the function that looks fairly regular when aggregated. Countour boxplots aim instead to capture the variablity of the uncertainity by trading in descripting
statistics for a measure called band depth. Each ensemble members band depth is computed as sum of the probabilities that the observations in any given ensemble fall within the max-min envelope defined by any two other ensembles. The bands are then sorted by band depth such that the median is the ensemble with about 50\% of its members withen all envelopes formed by other bands (so most centered). Outer bands are chosen according to the task at hand. In \cite{Whitaker2013}, they apply the countour boxplots to temperature visualizations. %%insert figures here. 
The authors argue that countour boxplots are an improved visual idiom over the traditional spagehttie plots seen in (fig) because as seen in (fig), the spagehetti plots get nosy as the number of plots increase and so it's hard to tease out specific patterns. Instead in (fig3), the authors remove most of the bands and instead visualize an outliser envelope(light gray) and a more central envelope(dark gray). It retains much of the information of the spagethhit, but removes the visual noise of the lines.  
